%%%%%
%Soubor: dokumentace.tex
%Datum: 31.3.2013
%Autor: Jan Wrona, xwrona00@stud.fit.vutbr.cz
%Projekt: Dokumentace k projektu 2 pro predmet IPK
%%%%%

\documentclass[a4paper, 11pt]{article}[31.03.2013]
  \usepackage[czech]{babel}
  \usepackage[utf8]{inputenc}
  \usepackage[T1]{fontenc}
  \usepackage[text={18cm, 26cm}, left=1.5cm, top=2cm]{geometry}

\begin{document}
\noindent
Dokumentace k projektu 2 do IPK 2012/2013\\
Název projektu: Program pro vyhledání informací o uživatelích Unixového OS\\
Jméno a příjmení: Jan Wrona\\
Login: xwrona00

\subsection*{Popis aplikačního protokolu}
Mým cílem při tvorbě aplikačního protokolu byla jeho jednoduchost, spolehlivost,
snadnost implementace jak na straně klienta tak serveru. Proto se jedná o textový
protokol, který je (po vypsání) čitelný a pochopitelný nejen strojově. Taktéž jsem
se snažil přenést většinu operací na serverový program, proto klient po přijetí zprávy
už nemusí vykonávat žádné operace, pouze zobrazení na standardní výstup.

Kompletní komunikace mezi serverem a klientem je zajištěna pomocí dvou zpráv.
První z nich vytvoří klient podle zadaných parametrů, připojí se na server na specifikovaném portu
a textovou zprávu (požadavek) odešle. Následně čeká na odpověd. Server je konkurentní, 
implementovaný s využitím procesů, a po přijetí spojení vytvoří nový proces. Rodičovský
proces se vrací zpět k čekání na další spojení, přičemž vytvořený proces se s přijatým
spojením vypořádá. Nejprve přijme příchozí požadavek, zpracuje jej a vytvoří podle něj
odpověď. Zpráva s odpovědí je odeslána, čímž je práce vytvořeného procesu ukončena.
V tuto chvíli čekající klient tuto odpověď přijme a zobrazí.
\subsection*{Formát požadavku}
Požadavek se skládá ze čtyř řádků, každý zakončen znakem LF (Line feed) s ordinární hodnotou
0x0A. První řádek obsahuje řetězec \texttt{msglen=length}, kde \texttt{length} je délka
(velikost) zprávy v bajtech bez samotného prvního řádku. Číslo je v šestnáctkové soustavě. Druhý řádek značí kritérium, 
podle kterého se bude vyhledávat a je ve formátu \texttt{criterion=login} nebo \texttt{criterion=uid}.
Následující řádek zahrnuje výčet zadaných argumentů, které se budou hledat. Řetězec je ve formátu
\texttt{arguments=list}, kde \texttt{list} je výčet argumentů oddělených čárkou (0x44).
Čárka je i za posledním argumentem a při nezadání žádných argumentů je \texttt{list} prázdný řetězec.
Posledním řádkem je řetězec \texttt{prArgs=args}, kde \texttt{args} je sekvence šesti
jednoznakových čísel značících, které položky požaduje klient vypsat. Tyto položky jsou
uživatelské jméno, UID, GID, gecos, domovský adresář a shell. Ve stejném pořadí jsou značeny
zmíněnou sekvencí čísel. Pokud položka vyžádaná není, číslu je nula, pokud vyžádaná je, číslo
je pořadí, ve kterém byl klientovi zadán argument na příkazové řádce tzn. pořadí, ve kterém
je výpis požadován.

\noindent
{\bf Příklad:}\\
spuštění programu:\\
\texttt{./client -h eva.fit.vutbr.cz -p 10000 -u 10 20 30 -L -G -S}\\
vygenerovaný požadavek: \\
\texttt{msglen=30\\
criterion=uid\\
arguments=10,20,30,\\
prArgs=102003\\}

\subsection*{Formát odpovědi}
Odpověď je stejně jako požadavek zahájena řádkem \texttt{msglen=length}, který má význam
i formát totožný s požadavkem. Následující řádky obsahují informace v textové podobě,
které klient požadoval. Jednotlivé položky v rámci řádku jsou odděleny mezerou.
Každá položka na samostatném řádku (ukončovač opět LF). Pokud server
nenalezl shodu, obsahuje řádek místo požadovaných informací chybové hlášení.

\noindent
{\bf Příklad:}\\
při požadavku uvedeném v předchozím příkladu může odpověď vypadat následovně:\\
\texttt{msglen=44\\
daemon 1 /sbin/nologin\\
operator 20 /bin/ksh\\
Error: unknown UID '30'}
\end{document}
